\documentclass{article}

\input{template/packages.tex}
%%% Define the homeworkProblem environment
\newcommand{\enterProblemHeader}[1]{
    \nobreak\extramarks{}{Question \arabic{#1} continued on next page\ldots}\nobreak{}
    \nobreak\extramarks{Question \arabic{#1} (continued)}{Question \arabic{#1} continued on next page\ldots}\nobreak{}
}

\newcommand{\exitProblemHeader}[1]{
    \nobreak\extramarks{Question \arabic{#1} (continued)}{Question \arabic{#1} continued on next page\ldots}\nobreak{}
    \stepcounter{#1}
    \nobreak\extramarks{Question \arabic{#1}}{}\nobreak{}
}

\setcounter{secnumdepth}{0}
\newcounter{partCounter}
\newcounter{homeworkProblemCounter}
\setcounter{homeworkProblemCounter}{1}
\nobreak\extramarks{Question \arabic{homeworkProblemCounter}}{}\nobreak{}

\newenvironment{homeworkProblem}[1][-1]{
    \ifnum#1>0
        \setcounter{homeworkProblemCounter}{#1}
    \fi
    \section{Question \arabic{homeworkProblemCounter}}
    \setcounter{partCounter}{1}
    \enterProblemHeader{homeworkProblemCounter}
}{
    \exitProblemHeader{homeworkProblemCounter}
}

\input{template/macros.tex}
\input{template/homework/settings.tex}

\title{
    \vspace{2in}
        \textmd{\textbf{MT232P - Analysis}}\\
        % \textmd{\textbf{MT241P - Finite Mathematics}}\\
        % \textmd{\textbf{MT251P - Foundations of Euclidean Geometry}}\\
    \vspace{1in}
    \textmd{\textbf{Assignment \#4}}\\
    \vspace{1in}
}

\author{
    \hmwkAuthorName\\
    \hmwkStudentnum
}

\date{}

\begin{document}

\maketitle

\pagebreak

\begin{homeworkProblem}
    \homBox{
        Let $x_1 = 1$, and set $x_{n+1} = \sqrt{2 + x_n}$ for all $n \in \mathbb{N}$. Use the Monotone Convergence
        Theorem to show that $\left\{x_n\right\}_1^\infty$ converges, and find $\lim_{n \to \infty} x_n$.
    }
    \solution
    First, we show that the sequence $\mset{x_n}$ is increasing. Since $x_1 = 1$, we have $x_{n+1} = \sqrt{2+x_n} > \sqrt{2+1} = \sqrt{3} > 1 = x_1$ for all $n \in \mathbb{N}$. Thus, the sequence is increasing.
    $\mset{x_n}$
    Next, we need to show that the sequence is bounded above. Note that $x_{n+1} = \sqrt{2+x_n} \le \sqrt{2+2} = \sqrt{4} = 2$ for all $n \in \mathbb{N}$. Thus, the sequence is bounded above.

    Since the sequence is increasing and bounded above, by the Monotone Convergence Theorem, the sequence $\mset{x_n}$ converges. To find the limit, we take the limit of both sides of the equation $x_{n+1} = \sqrt{2+x_n}$ as $n \to \infty$:

    $$\lim_{n \to \infty} x_{n+1} = \lim_{n \to \infty} \sqrt{2+x_n}$$

    Since the sequence $\mset{x_n}$ converges, the limit on the right-hand side exists and is equal to the limit of the sequence, say $L$. We can then write:

    $$L = \lim_{n \to \infty} \sqrt{2+x_n} = \sqrt{2+L}$$

    Squaring both sides, we get $L^2 = 2 + L$, so $L^2 - L - 2 = 0$. This quadratic equation has two solutions, $L = \frac{1 \pm \sqrt{5}}{2}$. However, since $L$ is the limit of an increasing sequence, it must be greater than or equal to the first term of the sequence, which is $1$. This implies that $L = \frac{1 + \sqrt{5}}{2}$.

    Therefore, the limit of the sequence $\mset{x_n}$ is $\frac{1 + \sqrt{5}}{2}$.
\end{homeworkProblem}
\begin{homeworkProblem}
    \homBox{
        Show that if every subsequence of $\mset{a_n}_1^\infty$ has itself a subsequence which converges to
        0, then $\mset{a_n}_1^\infty$ converges to 0
    }
    \solution
\end{homeworkProblem}
\begin{homeworkProblem}
    \homBox{
        Assume $\mset{a_n}_1^\infty$ and $\mset{b_n}_1^\infty$ are Cauchy sequences. Use a triangle inequality argument to
prove directly from the definition of a Cauchy sequence that $\mset{c_n}_1^\infty$, where $c_n = |an-bn|$,is also a Cauchy sequence.
    }
    \solution
\end{homeworkProblem}
\begin{homeworkProblem}
    \homBox{
        What is the value of $\lim_{x \to 1} \frac{1}{1+ \sqrt{x}}$? Prove your assertion using an $\epsilon - \delta$ argument.
    }
    \solution
    
\end{homeworkProblem}
\end{document}
% \begin{homeworkProblem}
%     \homBox{Hello!}
%     \part
%     \solution
%     \begin{align}
%         Hello
%     \end{align}
%     \begin{align*}
%         Hello
%     \end{align*}
%     \begin{align}
%         Hello
%     \end{align}
%     Hello
%     \part
%     Hello\\Hello\\Hello
%     \solution
%     Hello
% \end{homeworkProblem}


% \begin{homeworkProblem}
%     \defBox{Hello}{
% \begin{itemize}
%     \item It will rain today or tomorrow.
%     \item It will rain today and tomorrow.
%     \item It will rain today but not tomorrow.
%     \item It will rain today or tomorrow, but not both days.
% \end{itemize}
%     }
%     \part
%     \solution

%     \part
%     \solution
% \end{homeworkProblem}


% \begin{homeworkProblem}
%     \lemBox{Hello}{

%     }
% \end{homeworkProblem}


% \begin{homeworkProblem}
%     \thmBox{Hello}{
%         \tcblower
%         Hello
%     }
% \end{homeworkProblem}