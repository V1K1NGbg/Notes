\documentclass{article}

%%% The following looks horrible, but essentially sets up biblatex for producing
%%% bibliographies that look nice (you likely will not even need this)
\usepackage[style=authoryear-comp, maxcitenames=2, isbn=false, url=false,
  giveninits=true, doi=false, eprint=false, dashed=false, date=year,
  related=false, mergedate=true]{biblatex}
\renewbibmacro{in:}{%
  \ifboolexpr{%
    test {\ifentrytype{article}}%
    or
    test {\ifentrytype{inproceedings}}%
  }{}{\printtext{\bibstring{in}\intitlepunct}}%
}
\renewbibmacro*{cite:labelyear+extrayear}{%
  \ifentrytype{online}
  {}
  {\iffieldundef{labelyear}
    {}
    {\printtext[bibhyperref]{%
        \printfield{labelyear}
        \printfield{extrayear}}}}}

\AtEveryBibitem{%
  \clearfield{pagetotal}%
}

\DeclareFieldFormat[article,inbook,incollection,techreport,inproceedings,patent,thesis,unpublished]{title}{#1\isdot}
\DeclareFieldFormat{pages}{#1}

\usepackage[plain]{algorithm} % algorithms package
\usepackage{algpseudocode} % pseudo-code package
\usepackage{amsbsy} % for producing bold maths symbols
\usepackage{amsfonts} % an extended set of fonts for maths
\usepackage{amssymb} % various maths symbols
\usepackage{amsthm} % for producing theorem-like environments
\usepackage{datetime2} % managing dates and times
\usepackage{delimseasy} % makes easy the manual sizing of brackets, square brackets, and curly brackets
\usepackage{enumitem} % customing list environments
\usepackage{extramarks} % extra marks
\usepackage{fancyhdr} % headers and footers
\usepackage{float} % makes dealing with floats (e.g. tables and figures) easier
\usepackage{framed} % for producing framed boxes
\usepackage{graphicx} % for including graphics in the document
\usepackage{hyperref} % automatically produce hyperlinks for cross-references
\usepackage{import} % Import and subimport
\usepackage{listings} % Code blocks
\usepackage{mathtools} % package for maths (fixes some deficiences of amsmath so is preferred)
\usepackage{mdframed} % boxes
\usepackage{microtype} % better font sizing (extremely helpful with long equations!)
\usepackage{newtx} % a fonts package
\usepackage{pdfpages} % for including pdf documents inside the compiled pdf
\usepackage{pgf} % produce pdf graphics using LaTeX
\usepackage{pgfplots} % create normal/logarithmic plots in two and three dimensions
\pgfplotsset{compat=1.18} % sorts out the compatability warning
\usepackage{physics} % useful for vector calculus and linear algebra symbols
\usepackage[most]{tcolorbox} % for producing coloured boxes
\tcbuselibrary{theorems} % theorems with tcolorbox
\usepackage{tikz-3dplot} % for producing 3d plots
\usepackage{tikz} % for drawing graphics in LaTeX
\usepackage{tkz-base} % drawing with a Cartesian coordinate system
\usepackage{tkz-euclide} % drawing in Euclidean geometry
\usepackage{xcolor} % a package for colours

\usetikzlibrary{automata, positioning}

\DeclareMathAlphabet{\mathcal}{OMS}{cmsy}{m}{n}

%%% Define the homeworkProblem environment
\newcommand{\enterProblemHeader}[1]{
    \nobreak\extramarks{}{Problem \arabic{#1} continued on next page\ldots}\nobreak{}
    \nobreak\extramarks{Problem \arabic{#1} (continued)}{Problem \arabic{#1} continued on next page\ldots}\nobreak{}
}

\newcommand{\exitProblemHeader}[1]{
    \nobreak\extramarks{Problem \arabic{#1} (continued)}{Problem \arabic{#1} continued on next page\ldots}\nobreak{}
    \stepcounter{#1}
    \nobreak\extramarks{Problem \arabic{#1}}{}\nobreak{}
}

\setcounter{secnumdepth}{0}
\newcounter{partCounter}
\newcounter{homeworkProblemCounter}
\setcounter{homeworkProblemCounter}{1}
\nobreak\extramarks{Problem \arabic{homeworkProblemCounter}}{}\nobreak{}

\newenvironment{homeworkProblem}[1][-1]{
    \ifnum#1>0
        \setcounter{homeworkProblemCounter}{#1}
    \fi
    \section{Problem \arabic{homeworkProblemCounter}}
    \setcounter{partCounter}{1}
    \enterProblemHeader{homeworkProblemCounter}
}{
    \exitProblemHeader{homeworkProblemCounter}
}

\renewcommand{\part}{\textbf{\\\\\large Part \Alph{partCounter}}\stepcounter{partCounter}\\} % part macro
\newcommand{\solution}{\textbf{\\\large Solution}\\} % solution macro

\newmdenv[
    backgroundcolor=gray!20,
    skipabove=\topsep,
    skipbelow=\topsep,
]{grayBoxed}

% General writing
\newcommand{\bracket}[1]{\left(#1\right)} % for automatic resizing of brackets
\newcommand{\sbracket}[1]{\left[#1\right]} % for automatic resizing of square brackets
\newcommand{\mset}[1]{\left\{#1\right\}} % for automatic resizing of curly brackets
\newcommand{\defeq}{\coloneqq} % the "defined as" command
\newcommand{\RNum}[1]{\uppercase\expandafter{\romannumeral #1\relax}} % uppercase roman numerals

\newcommand{\rednote}[1]{{\color{red} #1}} % for red text
\newcommand{\bluenote}[1]{{\color{blue} #1}} % for blue text
\newcommand{\greennote}[1]{{\color{green} #1}} % for green text

% Blackboard Maths Symbols
\newcommand{\N}{\mathbb{N}} % natural numbers
\newcommand{\Q}{\mathbb{Q}} % rational numbers
\newcommand{\Z}{\mathbb{Z}} % integers
\newcommand{\R}{\mathbb{R}} % real numbers
\newcommand{\C}{\mathbb{C}} % complex numbers
\newcommand{\E}{\mathbb{E}} % expectation operators

% Aesthetic

\newcommand{\defBox}[2]{
    \begin{tcolorbox}[colback=cyan!5!white,colframe=cyan!40!black,
        colbacktitle=cyan!40!black,title=Definition: #1]
        #2
    \end{tcolorbox}
}

\newcommand{\lemBox}[2]{
    \begin{tcolorbox}[colback=cyan!5!white,colframe=cyan!50!black,
        colbacktitle=cyan!50!black,title=Lemma: #1]
        #2
    \end{tcolorbox}
}

\newcommand{\thmBox}[2]{
    \begin{tcolorbox}[colback=cyan!5!white,colframe=cyan!70!black,
        colbacktitle=cyan!70!black,title=Theorem: #1]
        #2
    \end{tcolorbox}
}

% Homework Specific
\newcommand{\homBox}[1]{
    \begin{tcolorbox}[colback=cyan!5!white,colframe=cyan!50!black]
        #1
    \end{tcolorbox}
}
%%% DOCUMENT SETTINGS
\topmargin=-0.45in
\evensidemargin=0in
\oddsidemargin=0in
\textwidth=6.5in
\textheight=9.0in
\headsep=0.25in
\linespread{1.1}
\pagestyle{fancy}
\lhead{\hmwkAuthorName}
\chead{}
\rhead{\hmwkStudentnum}
\lfoot{\lastxmark}
\cfoot{\thepage}
\renewcommand\headrulewidth{0.4pt}
\renewcommand\footrulewidth{0.4pt}
\setlength\parindent{0pt}


%%% HOMEWORK DETAILS
\newcommand{\hmwkStudentnum}{22337763}
\newcommand{\hmwkClassInstructor}{LECTURER NAME AND TITLE}
\newcommand{\hmwkAuthorName}{\textbf{Viktor Ilchev}}


\title{
    \vspace{2in}
        \textmd{\textbf{MT232P - Analysis}}\\
        % \textmd{\textbf{MT241P - Finite Mathematics}}\\
        % \textmd{\textbf{MT251P - Foundations of Euclidean Geometry}}\\
    \vspace{1in}
    \textmd{\textbf{Assignment \#4}}\\
    \vspace{1in}
}

\author{
    \hmwkAuthorName\\
    \hmwkStudentnum
}

\date{}

\begin{document}

\maketitle

\pagebreak

\begin{homeworkProblem}
    \homBox{
        Let $x_1 = 1$, and set $x_{n+1} = \sqrt{2 + x_n}$ for all $n \in \mathbb{N}$. Use the Monotone Convergence
        Theorem to show that $\left\{x_n\right\}_1^\infty$ converges, and find $\lim_{n \to \infty} x_n$.
    }
    \solution
    To use the Monotone Convergence Theorem, a sequence must be monotone. For a sequence to be monotone, we need to show that it is increasing and bounded. To show that the sequence is bounded we can use induction.\\Base case: $$x_2 = \sqrt{2+1} > x_1 = 1$$ $$\sqrt{3} > 1$$ Assume: $x_{n + 2} > x_{n + 1}$\\Then: $$x_{n+3} > x_{n+2}$$ $$\sqrt{2 + x_{n + 2}} > \sqrt{2 + x_{n + 1}}$$ $$2 + x_{n + 2} > 2 + x_{n + 1}$$ $$x_{n + 2} > x_{n + 1} \forall n \in \N$$ Because ${x_n}$ is strictly increasing, we know that if we prove it is also bounded, then we will know that it is monotone.\\ For a sequence to be bounded, $\exists M \in \N$ such that $|x_n| < M \forall n \in \N$\\ We can show that $x_n < 2 \forall n \in \N$ by induction\\Base case: $$n = 1, x_n = 1 < 2$$Assume: $$x_n <2$$, then we must show that $$x_{n+1} < 2$$ $$\sqrt{2 + x_n} < 2$$ $$2 + x_n < 4$$ $$x_n < 2$$, which is true by our assumption. Therefore since the sequence is increasing and bounded it is also monotone.Then by using the Monotone Convergence Theorem we know that ${x_n}$ converges.\\ Since ${x_n}$ converges, we know that $\lim_{n \to \infty}{x_n}$ exists.
    Let L = $\lim_{n \to \infty}{x_n}$, then since $x_{n+1} = \sqrt{2 + x_n}$: $$x_{n+1}^2 = 2 + x_n$$ $$\lim_{n \to \infty}{(x_{n + 1})}^2 = L^2$$ $$\lim_{n \to \infty}{2 + x_n} = 2 + L$$, because ($x_{n + 1}^2 = 2 + x_n$), we know that $L^2 = 2 + L$. Then to find L we use the -b formula to get $L_1 = 2$ and $L_2 = -1$, however since ${x_n}$ is increasing and $x_1 = 1$, then the sequence can't converge to -1, therefore ${x_n}$ converges to 2
\end{homeworkProblem}
\begin{homeworkProblem}
    \homBox{
        Show that if every subsequence of $\mset{a_n}_1^\infty$ has itself a subsequence which converges to
        0, then $\mset{a_n}_1^\infty$ converges to 0
    }
    \solution
    Let ${a_{n_k}}$ be a subsequence of ${a_n}$. We know that ${a_{n_k}}$ has a subsequence of it's own ${a_{n_{k_j}}}$. This new subsequence also has to be a subsequence of ${a_n}$\\Assume ${a_n}$ doesn't  converge to 0. We know that some subsequences of ${a_n}$ converge to 0, meaning that if ${a_n}$ doesn't converge to 0, then it must also have a subsequence which also doesn't converge to 0. Let's call this subsequence ${a_{n_s}}$. If we show that this subsequence also converges to 0, then ${a_{n}}$ must also converge to 0. Because ${a_{n_s}}$  is a subsequence of ${a_{n}}$, then it has a subsequence convergent to 0. By assumption, ${a_{n_s}}$ doesn't converge to 0. Therefore it must have a subsequence which doesn't converge to 0. Call this ${a_{n_{s_k}}}$. This is also a subsequence of ${a_n}$. Then it must have a subsequence that converges to 0  even tough that does not.\\If we repeat the steps above, eventually we will get that the subsequence must converge to 0.
\end{homeworkProblem}
\begin{homeworkProblem}
    \homBox{
        Assume $\mset{a_n}_1^\infty$ and $\mset{b_n}_1^\infty$ are Cauchy sequences. Use a triangle inequality argument to
prove directly from the definition of a Cauchy sequence that $\mset{c_n}_1^\infty$, where $c_n = |a_n-b_n|$,is also a Cauchy sequence.
    }
    \solution
    Using the triangle inequality to prove that ${c_n}_{n \in \N}$ is also a Cauchy sequence, where $c_n = |a_n - b_n|$. To show that ${c_n}$ is a Cauchy sequence, we need to show that $\forall \varepsilon > 0$, $\exists N \in \N$ such that $|c_n - c_m| < \varepsilon$, where $n, m > N$. We know that $\forall \varepsilon > 0, \exists N$ such that $m, n > N$. Then: $$|a_n - a_m| < \frac{\varepsilon}{2}, |b_n - b_m| < \frac{\varepsilon}{2}$$
    We can show this as: $$|a_{n_1} - a_{m_1}| < \frac{\varepsilon}{2}, |b_{n_1} - b_{m_1}| < \frac{\varepsilon}{2}$$if $m_1, n_1 > N_1$ and $$|a_{n_2} - a_{m_2}| < \frac{\varepsilon}{2}, |b_{n_2} - b_{m_2}| < \frac{\varepsilon}{2}$$ if $m_2, n_2 > N_2$. Let $N = \max \mset{N_1, N_2}$, then $$|a_n - a_m| < \frac{\varepsilon}{2}, |b_n - b_m| < \frac{\varepsilon}{2}$$ if $m, n > N$ Therefore we can  show the following: If $n_1$: $$|a - b| \geq ||a| - |b||$$(The Triangle Inequality) $$|c_n - c_m| = ||a_n - b_n| - |a_m - b_m||$$ $$\leq |a_n - b_n - a_m + b_m|$$ $$\leq|a_n - a_m - (b_n - b_m)|$$ $$\leq|a_n - a_m| + |b_n - b_m|$$ $$< \frac{\varepsilon}{2} + \frac{\varepsilon}{2}$$ $$< \varepsilon$$ if n, m > N
\end{homeworkProblem}
\begin{homeworkProblem}
    \homBox{
        What is the value of $\lim_{x \to 1} \frac{1}{1+ \sqrt{x}}$? Prove your assertion using an $\epsilon - \delta$ argument.
    }
    \solution
    Let $f(x) = \frac{1}{1 + \sqrt{x}}$. Let $h(x) = 1 + \sqrt{x}$, and $g(x) = 1$, then $\frac{g}{h}(x) = f(x)$. Therefore $\frac{\lim_{x \to 1}{g(x)}}{\lim_{x \to 1}{h(x)}} = \lim_{x \to 1}{f(x)}$. The limit of $g(x) = 1 \forall x$, The limit of $h(x)$ we can guess that is $h(1) = 2$ from the formula. To prove it, we must show that $\forall \varepsilon > 0, \exists \delta$ such that if $0 < |x - 1| < \delta$. Then: $$|h(x) - 2| < \varepsilon$$ $$|\sqrt{x} - 1| < \varepsilon$$ $$-\varepsilon < \sqrt{x} - 1 < \varepsilon$$ $$(1-\varepsilon)^2 < x <(1+\varepsilon)^2$$ $$(1-\varepsilon)^2 -1 < x - 1 <(1+\varepsilon)^2 -1$$ $$|x - 1| < \delta$$ and $$|x  - 1| < \max \mset{|(1-\varepsilon)^2 -1|, |(1+\varepsilon)^2 - 1}$$ Then $\delta = \max \mset{|(1-\varepsilon)^2 -1|, |(1+\varepsilon)^2 - 1}$\\ Therefore we have proven $\lim_{x \to 1}{h(x) = 2}$.\\ Therefore since $\lim_{x \to 1}{h(x) = 2}$ and $\lim_{x \to 1}{g(x) = 1}$  and $\lim_{x \to 1}{f(x)} = \frac{\lim_{x \to 1}{g(x)}}{\lim_{x \to 1}{h(x)}}$. Then by Theorem 3.10.4, $\lim_{x \to 1}{f(x)} = \frac{\lim_{x \to 1}{g(x)}}{\lim_{x \to 1}{h(x)}} = \frac{1}{2}$
\end{homeworkProblem}
\end{document}
% \begin{homeworkProblem}
%     \homBox{Hello!}
%     \part
%     \solution
%     \begin{align}
%         Hello
%     \end{align}
%     \begin{align*}
%         Hello
%     \end{align*}
%     \begin{align}
%         Hello
%     \end{align}
%     Hello
%     \part
%     Hello\\Hello\\Hello
%     \solution
%     Hello
% \end{homeworkProblem}


% \begin{homeworkProblem}
%     \defBox{Hello}{
% \begin{itemize}
%     \item It will rain today or tomorrow.
%     \item It will rain today and tomorrow.
%     \item It will rain today but not tomorrow.
%     \item It will rain today or tomorrow, but not both days.
% \end{itemize}
%     }
%     \part
%     \solution

%     \part
%     \solution
% \end{homeworkProblem}


% \begin{homeworkProblem}
%     \lemBox{Hello}{

%     }
% \end{homeworkProblem}


% \begin{homeworkProblem}
%     \thmBox{Hello}{
%         \tcblower
%         Hello
%     }
% \end{homeworkProblem}