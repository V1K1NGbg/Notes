\documentclass{article}

\input{template/packages.tex}
%%% Define the homeworkProblem environment
\newcommand{\enterProblemHeader}[1]{
    \nobreak\extramarks{}{Question \arabic{#1} continued on next page\ldots}\nobreak{}
    \nobreak\extramarks{Question \arabic{#1} (continued)}{Question \arabic{#1} continued on next page\ldots}\nobreak{}
}

\newcommand{\exitProblemHeader}[1]{
    \nobreak\extramarks{Question \arabic{#1} (continued)}{Question \arabic{#1} continued on next page\ldots}\nobreak{}
    \stepcounter{#1}
    \nobreak\extramarks{Question \arabic{#1}}{}\nobreak{}
}

\setcounter{secnumdepth}{0}
\newcounter{partCounter}
\newcounter{homeworkProblemCounter}
\setcounter{homeworkProblemCounter}{1}
\nobreak\extramarks{Question \arabic{homeworkProblemCounter}}{}\nobreak{}

\newenvironment{homeworkProblem}[1][-1]{
    \ifnum#1>0
        \setcounter{homeworkProblemCounter}{#1}
    \fi
    \section{Question \arabic{homeworkProblemCounter}}
    \setcounter{partCounter}{1}
    \enterProblemHeader{homeworkProblemCounter}
}{
    \exitProblemHeader{homeworkProblemCounter}
}

\input{template/macros.tex}
\input{template/homework/settings.tex}

\title{
    \vspace{2in}
        % \textmd{\textbf{MT232P - Analysis}}\\
        \textmd{\textbf{MT241P - Finite Mathematics}}\\
        % \textmd{\textbf{MT251P - Foundations of Euclidean Geometry}}\\
    \vspace{1in}
    \textmd{\textbf{Assignment \#4}}\\
    \vspace{1in}
}

\author{
    \hmwkAuthorName\\
    \hmwkStudentnum
}

\date{}

\begin{document}

\maketitle

\pagebreak

\begin{homeworkProblem}
    \homBox{
        Suppose Met Eirann provides the following predictions:\\
        (P1) There is a 60 percent chance that it will rain today.\\
        (P2) There is a 50 percent chance that it will rain tomorrow.
    }
    \part
    \homBox{
        Let 'R' indicate a day with rain and 'N' a day with no rain. List an appro-
        priate sample space $\Omega$.
    }
    \solution
    $$\Omega = \mset{RR, RN, NR, NN}$$
    \part
    \homBox{
        Let A be the event that it rains today and let B be the event that it rains
        tomorrow. List the outcomes of the following events:
        \begin{itemize}
            \item $A^c$
            \item $A\cup B$
            \item $A \cap B$
            \item $A \cap B^c$
            \item $(A \cup B)^c$
        \end{itemize}
    }
    \solution
    $$A^c = \mset{NR, NN}$$
    $$A \cup B = \mset{RR, RN, NR}$$
    $$A \cap B = \mset{RR}$$
    $$A \cap B^c = \mset{RN}$$
    $$(A \cup B)^c = \mset{NN}$$
    \pagebreak
    \part
    \homBox{
        Find the probabilities for the following events:
        \begin{itemize}
            \item It will rain today or tomorrow.
            \item It will rain today and tomorrow.
            \item It will rain today but not tomorrow.
            \item It will rain today or tomorrow, but not both days.
        \end{itemize}
    }
    \solution
    "It will rain today or tomorrow." is the same as $A \cup B$,and therefore form part A: $$P(A \cup B) = P(RR) + P(RN) + P(NR) =$$$$= P(A)P(B) + P(A)P(B^c) + P(A^c)P(B) =$$$$= 0.6*0.5 + 0.6*0.5 + 0.4*0.5 = 0.8$$
    "It will rain today and tomorrow." is the same as $A \cap B$,and therefore form part A: $$P(A \cap B) = P(RR)=$$$$= P(A)P(B)=$$$$= 0.6*0.5 = 0.3$$
    "It will rain today but not tomorrow." is the same as $A \cap B^c$,and therefore form part A: $$P(A \cap B^c) = P(RN) =$$$$= P(A)P(B^c) =$$$$= 0.6*0.5 = 0.3$$
    "It will rain today or tomorrow, but not both days." has the following outcomes: RN, NR. Then The probability for that event (Let's call it C) will be: $$P(C) = P(RN) + P(NR) =$$$$= P(A)P(B^c) + P(A^c)P(B) =$$$$= 0.6*0.5 + 0.4*0.5 = 0.5$$

\end{homeworkProblem}
\begin{homeworkProblem}
    \homBox{
        Assume you flip a fair coin with a friend n times, where n $\geq$ 1
    }
    \part
    \homBox{
        How many possible outcomes are there?
    }
    \solution
    There are $2^n$ solutions since on each throw only one of two things can happen. The coin will either fall on heads or on tails.
    \part
    \homBox{
        What is the probability of each outcome?
    }
    \solution
    Since the coin is a fair coin, then the probability of heads is the same as the probability of tails, which is $(\frac{1}{2})^n$, where n is the number of throws.
    \part
    \homBox{
        If you throw the coin ten times, what is the chance of there being exactly
        one tail in any three consecutive throws?
    }
    \solution
    If we represent a tail with a T and a head with an H, then we can have the following configurations for three consecutive throws that satisfy the prompt:
    \begin{itemize}
        \item HHT
        \item HTH
        \item THH
    \end{itemize}

    \homBox{
        Next you play a game for money. Each time heads comes up you win a Euro, each
        time tails comes up you lose a Euro. However, as soon as you lose for the first time
        you claim you have to go home and stop playing.
    }
    \part
    \homBox{
        Describe the sample space Ω in terms of your possible wins/losses.
    }
    \solution
    $$\Omega = \mset{T, HT, HHT, HHHT, HHHHT, HHHHHT, \dots}$$
    \part
    \homBox{
        For each outcome $\omega \in \Omega$ give its probability.
    }
    \solution
    $\omega = (\frac{1}{2})^n$, where n is the index of the element in $\Omega$
    \part
    \homBox{
        What is the probability of you winning at least one, but less than four Euro?
    }
    \solution
    The probability of winning at least 1, but less than 4 euro is the same as the probability of getting 1, 2, or 3 Euro, which is the same as the probability of getting 1 Euro + the probability of getting 2 Euro + the probability of getting 3 Euro, which is the same as: $$P(HHT) + P(HHHT) + P(HHHHT) =$$$$= P(H)P(H)P(T) + P(H)P(H)P(H)P(T) + P(H)P(H)P(H)P(H)P(T) =$$$$= \frac{1}{2}\frac{1}{2}\frac{1}{2} + \frac{1}{2}\frac{1}{2}\frac{1}{2}\frac{1}{2} + \frac{1}{2}\frac{1}{2}\frac{1}{2}\frac{1}{2}\frac{1}{2} =$$$$= \frac{1}{2^3} + \frac{1}{2^4} + \frac{1}{2^5} =$$$$= \frac{2^2}{2^5} + \frac{2}{2^5} + \frac{1}{2^5} =$$$$= \frac{7}{2^5}$$
    \part
    \homBox{
        What is the probability of you winning more than two Euro?
    }
    \solution
    The probability of winning more than two Euro, is the same as winning 3 or more Euro, which is the same as getting at least 4 heads, allowing for the next throw to leave us with at least 3 euro even if the throw gives a tail. The probability of that is: $$P(HHHH)=$$$$=P(H)P(H)P(H)P(H)=$$$$=\frac{1}{2}\frac{1}{2}\frac{1}{2}\frac{1}{2}=$$$$=\frac{1}{16}$$
\end{homeworkProblem}
\begin{homeworkProblem}
    \homBox{
        You are at a party attended by k people, including you. What is the
        likelihood of somebody else at the party sharing your birthday? (We assume that
        nobody was born in a leap year). What is the likelihood if k = 23?
    }
    \solution
    k = 23 = 22 + 1(you)\\
    The chance for each person to have the same birthday as you is: $\frac{1}{365}$
    The chance for each person not to have the same birthday as you is: $\frac{364}{365}$
    Then the chance for no one to have the same birthday as you is: $(\frac{364}{365})^{22}$
\end{homeworkProblem}
\begin{homeworkProblem}
    \homBox{
        Use combinatorial arguments to prove that, for every integer n $\geq$ 0, $$\sum_{k=0}^n \binom{n}{k} = 2^n$$
    }
    \solution
    If we say n is a number of balls, the left side of the equation gives the possible subsets of n balls. As each ball can either be in the subset or not, then there must be $2^n$ possible subsets for n balls. Therefore we get: $\sum_{k=0}^n \binom{n}{k} = 2^n$
\end{homeworkProblem}
\begin{homeworkProblem}
    \homBox{
        Let A and B be finite sets such that A has n elements and B has
        m elements, where n $\geq$ m. How many injective functions $f : A \rightarrow B$ are there, that
        is, functions where $f(a_1) \neq f(a_2)$, whenever $a_1 \neq a_2$.
    }
    \solution
    If A has n elements, and B has m elements, and the function is injunctive, then:\\
    ${n}\choose{m}$ is the number of sets of size m possible to make with n elements\\
    $m!$ is the number of orientations you can have m elements in\\
    Therefore the amount of injective functions is ${n}\choose{m}$$m!$
\end{homeworkProblem}
\begin{homeworkProblem}
    \homBox{
        In how many ways can 2n tennis players be paired and assigned to
        n courts?
    }
    \solution
    Since there are 2n players and n courts with 2 players per court, then the possible pairs for the first court are ${2n}\choose{2}$. Since we already picked 2 people, then the possible pairs for the second court are ${2(n-1)}\choose{2}$. With that knowledge we can deduce that the combinations are: $${2n!}\choose{(2!)^n}$$
\end{homeworkProblem}
\begin{homeworkProblem}
    \homBox{
        How many distinct integer solutions does the equation: $$x_1 + x_2 + x_3 + x_4 = 100$$ heve, if:
    }
    \part
    \homBox{
        $x_i \geq 0$, for all i = 1, 2, 3, 4
    }
    \solution

    \part
    \homBox{
        $x_i \geq i$, for all i = 1, 2, 3, 4
    }
    \solution
\end{homeworkProblem}
\begin{homeworkProblem}
    \homBox{
        In my home town, it rains one third of all days. Traffic is heavy
        on half of the rainy days and a quarter of the dry days. If it's rainy and the traffic
        is heavy, then I am bound to be late for work half of all days. A quarter of the days
        that I'm late, it is not rainy but the traffic is heavy. Whenever there is light traffic,
        I am twice as likely to be late on a rainy day, compared to dry days. I am late for
        work one quarter of all days.
    }
    \part
    \homBox{
        Draw the tree diagram, where the first stage gives rain / no rain, the second
        stage gives traffic / no traffic and the third stage gives late / not late.
    }
    \solution
    \part
    \homBox{
        What is my chance of being on time on a rainy day with light traffic?
    }
    \solution
    \part
    \homBox{
        What is my chance of being on time on a dry day?
    }
    \solution
    \part
    \homBox{
        Given I was late today, what is the chance of there having been light traffic?
    }
    \solution
    \part
    \homBox{
        Given I was on time today and there was light traffic, what is the chance of
        there having been rain?
    }
    \solution
\end{homeworkProblem}
\end{document}
% \begin{homeworkProblem}
%     \homBox{Hello!}
%     \part
%     \solution
%     \begin{align}
%         Hello
%     \end{align}
%     \begin{align*}
%         Hello
%     \end{align*}
%     \begin{align}
%         Hello
%     \end{align}
%     Hello
%     \part
%     Hello\\Hello\\Hello
%     \solution
%     Hello
% \end{homeworkProblem}


% \begin{homeworkProblem}
%     \defBox{Hello}{
% \begin{itemize}
%     \item It will rain today or tomorrow.
%     \item It will rain today and tomorrow.
%     \item It will rain today but not tomorrow.
%     \item It will rain today or tomorrow, but not both days.
% \end{itemize}
%     }
%     \part
%     \solution

%     \part
%     \solution
% \end{homeworkProblem}


% \begin{homeworkProblem}
%     \lemBox{Hello}{

%     }
% \end{homeworkProblem}


% \begin{homeworkProblem}
%     \thmBox{Hello}{
%         \tcblower
%         Hello
%     }
% \end{homeworkProblem}