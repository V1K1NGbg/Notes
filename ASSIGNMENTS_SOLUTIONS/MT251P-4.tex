\documentclass{article}

\input{template/packages.tex}
%%% Define the homeworkProblem environment
\newcommand{\enterProblemHeader}[1]{
    \nobreak\extramarks{}{Question \arabic{#1} continued on next page\ldots}\nobreak{}
    \nobreak\extramarks{Question \arabic{#1} (continued)}{Question \arabic{#1} continued on next page\ldots}\nobreak{}
}

\newcommand{\exitProblemHeader}[1]{
    \nobreak\extramarks{Question \arabic{#1} (continued)}{Question \arabic{#1} continued on next page\ldots}\nobreak{}
    \stepcounter{#1}
    \nobreak\extramarks{Question \arabic{#1}}{}\nobreak{}
}

\setcounter{secnumdepth}{0}
\newcounter{partCounter}
\newcounter{homeworkProblemCounter}
\setcounter{homeworkProblemCounter}{1}
\nobreak\extramarks{Question \arabic{homeworkProblemCounter}}{}\nobreak{}

\newenvironment{homeworkProblem}[1][-1]{
    \ifnum#1>0
        \setcounter{homeworkProblemCounter}{#1}
    \fi
    \section{Question \arabic{homeworkProblemCounter}}
    \setcounter{partCounter}{1}
    \enterProblemHeader{homeworkProblemCounter}
}{
    \exitProblemHeader{homeworkProblemCounter}
}

\input{template/macros.tex}
\input{template/homework/settings.tex}

\title{
    \vspace{2in}
        % \textmd{\textbf{MT232P - Analysis}}\\
        % \textmd{\textbf{MT241P - Finite Mathematics}}\\
        \textmd{\textbf{MT251P - Foundations of Euclidean Geometry}}\\
    \vspace{1in}
    \textmd{\textbf{Assignment \#4}}\\
    \vspace{1in}
}

\author{
    \hmwkAuthorName\\
    \hmwkStudentnum
}

\date{}

\begin{document}

\maketitle

\pagebreak
\begin{homeworkProblem}
    \homBox{
        In each case below, state whether the statement is true or false. Justify your answer in
        each case.
    }
    \part
    \homBox{
        There are infinitely many 4 $\times$ 4 matrices that are not invertible.
    }
    \solution
    Since a square matrix is invertible if and only if its determinant is non-zero. Since there are infinitely many possible values for the elements of a 4 $\times$ 4 matrix, there are also infinitely many matrices that have a determinant of zero ie there are infinite amount of matrices with a zero row and any matrix with a zero row has det = 0 and therefore not invertible. Therefore the statement is True
    \part
    \homBox{
        There is a 4 $\times$ 4 invertible matrix A such that $A^3 = 2A^2$ and $\det A = 2$.
    }
    \solution
    $$A^3 = 2A^2$$ $$A(AA) =There 2(AA)$$ Since A is invertible, then $A^{-1}$ exists. So: $$A(AA)(A^{-1}A^{-1}) = 2(AA)(A^{-1}A^{-1})$$ Since $AA^{-1} = I_4 = A^{-1}A$, then: $$A(AA^{-1})(AA^{-1}) = 2(AA^{-1})(AA^{-1})$$ We also know that $AA^{-1} = I_4$ since A is a $4\times 4$ matrix. So: $$A(I_4)(I_4) = 2(I_4)(I_4)$$ $$A = 2(I_4)$$ $$\det(A) = \det(2(I_4))$$ $$\det(A) = 16$$
    Since $16 \neq 2$, then the statement is False.
    \part
    \homBox{
        There is a 4 $\times$ 4 matrix A such that $A^2 =
            \begin{pmatrix}
                -1 & 0 & 0 & 0 \\
                0  & 1 & 0 & 0 \\
                0  & 0 & 1 & 0 \\
                0  & 0 & 0 & 1
            \end{pmatrix}$
    }
    \solution
    $$\det(A^2) = \det(A)\det(A) = \det(A)^2$$
    Since $\det(A) \in \R$, then $\det(A^2) \geq 0$

    However, $$\det(A^2) = -1\det\begin{pmatrix}
            1 & 0 & 0 \\
            0 & 1 & 0 \\
            0 & 0 & 1
        \end{pmatrix}$$
        $$= -1\det\begin{pmatrix}
            1 & 0 \\
            0 & 1 
        \end{pmatrix}$$
        $$= -1$$
    This gives a contradiction. Therefore the statement is False.
\end{homeworkProblem}
\begin{homeworkProblem}
    \part
    \homBox{
        Prove that $\det (A^{-1}BA)$ = $\det (B)$, for all n $\times$ n matrices A, B, where A is invertible
        and n > 1.
    }
    \solution
    Let $A$ be an invertible $n \times n$ matrix, and let $B$ be any $n \times n$ matrix. We have
    $\det (A^{-1}BA) = \det (A^{-1}) \det (B) \det (A) = \frac{1}{\det (A)} \det (B) \det (A) = \det (B)$, so $\det (A^{-1}BA) = \det (B)$
    \part
    \homBox{
        Suppose $\underline{a} = i + 2j - k$, $\underline{b} = i + 3j + k$ and $\underline{c} = 3i + 8j + 4k$. Find $||\underline{w}||^2$ if $\underline{w} \in \R^3$ such that $\underline{w}.\underline{a} = 3$, $\underline{w}.\underline{b} = 5$ and $\underline{w}.\underline{c} = 17$.
    }
    \solution
    Since $\underline{w}$ is in $\R^3$, it must be represented in the form of $\underline{w} = xi + yj + zk$ for some $x, y, z \in \R$.

    We have the following:
    $$ \underline{w}.\underline{a} = (xi + yj + zk).(i + 2j - k) = x + 2y - z = 3 $$
    $$ \underline{w}.\underline{b} = (xi + yj + zk).(i + 3j + k) = x + 3y + z = 5 $$
    $$ \underline{w}.\underline{c} = (xi + yj + zk).(3i + 8j + 4k) = 3x + 8y + 4z = 17 $$Now we can solve for $x, y, z$ using a matrix:
    $$\begin{pmatrix}
            1 & 2 & -1 & 3  \\
            1 & 3 & 1  & 5  \\
            3 & 8 & 4  & 17
        \end{pmatrix}$$
    Replace $R_2$ with $R_2 - R_1$
    $$\begin{pmatrix}
            1 & 2 & -1 & 3  \\
            0 & 1 & 2  & 2  \\
            3 & 8 & 4  & 17
        \end{pmatrix}$$
    Replace $R_3$ with $R_3 - 3R_1$
    $$\begin{pmatrix}
            1 & 2 & -1 & 3 \\
            0 & 1 & 2  & 2 \\
            0 & 2 & 7  & 8
        \end{pmatrix}$$
    Replace $R_3$ with $R_3 - 2R_2$
    $$\begin{pmatrix}
            1 & 2 & -1 & 3 \\
            0 & 1 & 2  & 2 \\
            0 & 0 & 3  & 4
        \end{pmatrix}$$
    Replace $R_1$ with $R_1 - 2R_2$
    $$\begin{pmatrix}
            1 & 0 & -5 & -1 \\
            0 & 1 & 2  & 2  \\
            0 & 0 & 3  & 4
        \end{pmatrix}$$
    Replace $R_3$ with $\frac{1}{3}R_3$
    $$\begin{pmatrix}
            1 & 0 & -5 & -1          \\
            0 & 1 & 2  & 2           \\
            0 & 0 & 1  & \frac{4}{3}
        \end{pmatrix}$$
    Replace $R_2$ with $R_2 - 2R_3$
    $$\begin{pmatrix}
            1 & 0 & -5 & -1           \\
            0 & 1 & 0  & -\frac{2}{3} \\
            0 & 0 & 1  & \frac{4}{3}
        \end{pmatrix}$$
    Replace $R_1$ with $R_1 - (-5)R_3$
    $$\begin{pmatrix}
            1 & 0 & 0 & \frac{17}{3} \\
            0 & 1 & 0 & -\frac{2}{3} \\
            0 & 0 & 1 & \frac{4}{3}
        \end{pmatrix}$$
    Then: $x = \frac{17}{3}, y = -\frac{2}{3}, z = \frac{4}{3}$\\
    Therefore $||\underline{w}|| = \sqrt{x^2 + y^2 + z^2} = \sqrt{\frac{17}{3}^2 - \frac{2}{3}^2 + \frac{4}{3}^2} = \frac{\sqrt{309}}{3}$\\
    Then: $||\underline{w}||^2 = (\frac{\sqrt{309}}{3})^2 = \frac{103}{3}$
\end{homeworkProblem}
\begin{homeworkProblem}
    \homBox{
        Find the solution set of the following system of linear equations:
        $$x_1 - 4x_2 + 3x_3 = 0$$
        $$2x_1 - 6x_2 + 10x_3 = 6$$
        $$x_1 - 2x_2 + 7x_3 = 5$$
    }
    \solution
    We can solve for $x_1, x_2, x_3$ using a matrix:
    $$\begin{pmatrix}
            1 & -4 & 3  & 0 \\
            2 & -6 & 10 & 6 \\
            1 & -2 & 7  & 5
        \end{pmatrix}$$
    Replace $R_2$ with $R_2 - 2R_1$
    $$\begin{pmatrix}
            1 & -4 & 3 & 0 \\
            0 & 2  & 4 & 6 \\
            1 & -2 & 7 & 5
        \end{pmatrix}$$
    Replace $R_3$ with $R_3 - R_1$
    $$\begin{pmatrix}
            1 & -4 & 3 & 0 \\
            0 & 2  & 4 & 6 \\
            0 & 2  & 4 & 5
        \end{pmatrix}$$
    Replace $R_3$ with $R_3 - R_2$
    $$\begin{pmatrix}
            1 & -4 & 3 & 0  \\
            0 & 2  & 4 & 6  \\
            0 & 0  & 0 & -1
        \end{pmatrix}$$
    Notice how the last row states: $0 = -1$, therefore the system has no solutions.
\end{homeworkProblem}
\begin{homeworkProblem}
    \homBox{
        Find the solution set of the following system of linear equations:
        $$4x - 6y + 8z = 8$$
        $$x + 2y - 5z = 2$$
        $$y + 4x - 6z = 8$$
    }
    \solution
    We can solve for $x, y, z$ using a matrix:
    $$\begin{pmatrix}
            4 & -6 & 8  & 8 \\
            1 & 2  & -5 & 2 \\
            4 & 1  & -6 & 8
        \end{pmatrix}$$
    Replace $R_1$ with $R_2$
    $$\begin{pmatrix}
            1 & 2  & -5 & 2 \\
            4 & -6 & 8  & 8 \\
            4 & 1  & -6 & 8
        \end{pmatrix}$$
    Replace $R_2$ with $R_2 - 4R_1$
    $$\begin{pmatrix}
            1 & 2   & -5 & 2 \\
            0 & -14 & 28 & 0 \\
            4 & 1   & -6 & 8
        \end{pmatrix}$$
    Replace $R_3$ with $R_3 - 4R_1$
    $$\begin{pmatrix}
            1 & 2   & -5 & 2 \\
            0 & -14 & 28 & 0 \\
            0 & -7  & 14 & 0
        \end{pmatrix}$$
    Replace $R_2$ with $\frac{1}{2}R_2$
    $$\begin{pmatrix}
            1 & 2  & -5 & 2 \\
            0 & -7 & 14 & 0 \\
            0 & -7 & 14 & 0
        \end{pmatrix}$$
    Replace $R_3$ with $R_3 - R_2$
    $$\begin{pmatrix}
            1 & 2  & -5 & 2 \\
            0 & -7 & 14 & 0 \\
            0 & 0  & 0  & 0
        \end{pmatrix}$$
    Replace $R_2$ with $-\frac{1}{7}R_2$
    $$\begin{pmatrix}
            1 & 2 & -5 & 2 \\
            0 & 1 & -2 & 0 \\
            0 & 0 & 0  & 0
        \end{pmatrix}$$
    Replace $R_1$ with $R_1 - 2R_2$
    $$\begin{pmatrix}
            1 & 0 & -1 & 2 \\
            0 & 1 & -2 & 0 \\
            0 & 0 & 0  & 0
        \end{pmatrix}$$
    Notice how the matrix is in RREF and the last row states $0 = 0$, therefore the system has infinitely many solutions.\\
    Then the system of linear equations is: $$x - z = 2$$ $$y - 2z = 0$$
    By Remark 4, x, y are leading variables, and z is a free variable. We say that z = s, where s can be any real number.
    Then we can represent x, y in terms of z: $$ x = 2 + z = 2 + s $$ $$y = 2z = 2s$$
    Therefore the solution set of the system of equations is:$$\mset{(2 + s, 2s, s):s \in \R}$$
\end{homeworkProblem}
\begin{homeworkProblem}
    \homBox{
        Find $A^{-1}$ if $A =
            \begin{pmatrix}
                0 & 1  & 2 \\
                1 & 0  & 3 \\
                4 & -3 & 8
            \end{pmatrix}$
    }
    \solution
    $$A = \begin{pmatrix}
            0 & 1  & 2 \\
            1 & 0  & 3 \\
            4 & -3 & 8
        \end{pmatrix}
        \begin{pmatrix}
            1 & 0 & 0 \\
            0 & 1 & 0 \\
            0 & 0 & 1
        \end{pmatrix} = I_3$$
    Replace $R_1$ with $R_2$ in both matrices
    $$\begin{pmatrix}
            1 & 0  & 3 \\
            0 & 1  & 2 \\
            4 & -3 & 8
        \end{pmatrix}
        \begin{pmatrix}
            0 & 1 & 0 \\
            1 & 0 & 0 \\
            0 & 0 & 1
        \end{pmatrix}$$
    Replace $R_3$ with $R_3 - 4R_1$ in both matrices
    $$\begin{pmatrix}
            1 & 0  & 3  \\
            0 & 1  & 2  \\
            0 & -3 & -4
        \end{pmatrix}
        \begin{pmatrix}
            0 & 1  & 0 \\
            1 & 0  & 0 \\
            0 & -4 & 1
        \end{pmatrix}$$
    Replace $R_3$ with $R_3 - (-3)R_2$ in both matrices
    $$\begin{pmatrix}
            1 & 0 & 3 \\
            0 & 1 & 2 \\
            0 & 0 & 2
        \end{pmatrix}
        \begin{pmatrix}
            0 & 1  & 0 \\
            1 & 0  & 0 \\
            3 & -4 & 1
        \end{pmatrix}$$
    Replace $R_3$ with $\frac{1}{2}R_3$ in both matrices
    $$\begin{pmatrix}
            1 & 0 & 3 \\
            0 & 1 & 2 \\
            0 & 0 & 1
        \end{pmatrix}
        \begin{pmatrix}
            0           & 1  & 0           \\
            1           & 0  & 0           \\
            \frac{3}{2} & -2 & \frac{1}{2}
        \end{pmatrix}$$
    Replace $R_2$ with $R_2 - 2R_3$ in both matrices
    $$\begin{pmatrix}
            1 & 0 & 3 \\
            0 & 1 & 0 \\
            0 & 0 & 1
        \end{pmatrix}
        \begin{pmatrix}
            0           & 1  & 0           \\
            -2          & 4  & -1          \\
            \frac{3}{2} & -2 & \frac{1}{2}
        \end{pmatrix}$$
    Replace $R_1$ with $R_1 - 3R_3$ in both matrices
    $$\begin{pmatrix}
            1 & 0 & 0 \\
            0 & 1 & 0 \\
            0 & 0 & 1
        \end{pmatrix}
        \begin{pmatrix}
            -\frac{9}{2} & 7  & -\frac{3}{2} \\
            -2           & 4  & -1           \\
            \frac{3}{2}  & -2 & \frac{1}{2}
        \end{pmatrix}$$
    Since A is in RREF, then:
    $$A^{-1} = \begin{pmatrix}
            -\frac{9}{2} & 7  & -\frac{3}{2} \\
            -2           & 4  & -1           \\
            \frac{3}{2}  & -2 & \frac{1}{2}
        \end{pmatrix}$$
\end{homeworkProblem}
\end{document}
% \begin{homeworkProblem}
%     \homBox{Hello!}
%     \part
%     \solution
%     \begin{align}
%         Hello
%     \end{align}
%     \begin{align*}
%         Hello
%     \end{align*}
%     \begin{align}
%         Hello
%     \end{align}
%     Hello
%     \part
%     Hello\\Hello\\Hello
%     \solution
%     Hello
% \end{homeworkProblem}


% \begin{homeworkProblem}
%     \defBox{Hello}{
% \begin{itemize}
%     \item It will rain today or tomorrow.
%     \item It will rain today and tomorrow.
%     \item It will rain today but not tomorrow.
%     \item It will rain today or tomorrow, but not both days.
% \end{itemize}
%     }
%     \part
%     \solution

%     \part
%     \solution
% \end{homeworkProblem}


% \begin{homeworkProblem}
%     \lemBox{Hello}{

%     }
% \end{homeworkProblem}


% \begin{homeworkProblem}
%     \thmBox{Hello}{
%         \tcblower
%         Hello
%     }
% \end{homeworkProblem}