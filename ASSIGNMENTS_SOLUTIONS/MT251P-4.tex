\documentclass{article}

%%% The following looks horrible, but essentially sets up biblatex for producing
%%% bibliographies that look nice (you likely will not even need this)
\usepackage[style=authoryear-comp, maxcitenames=2, isbn=false, url=false,
  giveninits=true, doi=false, eprint=false, dashed=false, date=year,
  related=false, mergedate=true]{biblatex}
\renewbibmacro{in:}{%
  \ifboolexpr{%
    test {\ifentrytype{article}}%
    or
    test {\ifentrytype{inproceedings}}%
  }{}{\printtext{\bibstring{in}\intitlepunct}}%
}
\renewbibmacro*{cite:labelyear+extrayear}{%
  \ifentrytype{online}
  {}
  {\iffieldundef{labelyear}
    {}
    {\printtext[bibhyperref]{%
        \printfield{labelyear}
        \printfield{extrayear}}}}}

\AtEveryBibitem{%
  \clearfield{pagetotal}%
}

\DeclareFieldFormat[article,inbook,incollection,techreport,inproceedings,patent,thesis,unpublished]{title}{#1\isdot}
\DeclareFieldFormat{pages}{#1}

\usepackage[plain]{algorithm} % algorithms package
\usepackage{algpseudocode} % pseudo-code package
\usepackage{amsbsy} % for producing bold maths symbols
\usepackage{amsfonts} % an extended set of fonts for maths
\usepackage{amssymb} % various maths symbols
\usepackage{amsthm} % for producing theorem-like environments
\usepackage{datetime2} % managing dates and times
\usepackage{delimseasy} % makes easy the manual sizing of brackets, square brackets, and curly brackets
\usepackage{enumitem} % customing list environments
\usepackage{extramarks} % extra marks
\usepackage{fancyhdr} % headers and footers
\usepackage{float} % makes dealing with floats (e.g. tables and figures) easier
\usepackage{framed} % for producing framed boxes
\usepackage{graphicx} % for including graphics in the document
\usepackage{hyperref} % automatically produce hyperlinks for cross-references
\usepackage{import} % Import and subimport
\usepackage{listings} % Code blocks
\usepackage{mathtools} % package for maths (fixes some deficiences of amsmath so is preferred)
\usepackage{mdframed} % boxes
\usepackage{microtype} % better font sizing (extremely helpful with long equations!)
\usepackage{newtx} % a fonts package
\usepackage{pdfpages} % for including pdf documents inside the compiled pdf
\usepackage{pgf} % produce pdf graphics using LaTeX
\usepackage{pgfplots} % create normal/logarithmic plots in two and three dimensions
\pgfplotsset{compat=1.18} % sorts out the compatability warning
\usepackage{physics} % useful for vector calculus and linear algebra symbols
\usepackage[most]{tcolorbox} % for producing coloured boxes
\tcbuselibrary{theorems} % theorems with tcolorbox
\usepackage{tikz-3dplot} % for producing 3d plots
\usepackage{tikz} % for drawing graphics in LaTeX
\usepackage{tkz-base} % drawing with a Cartesian coordinate system
\usepackage{tkz-euclide} % drawing in Euclidean geometry
\usepackage{xcolor} % a package for colours

\usetikzlibrary{automata, positioning}

\DeclareMathAlphabet{\mathcal}{OMS}{cmsy}{m}{n}

%%% Define the homeworkProblem environment
\newcommand{\enterProblemHeader}[1]{
    \nobreak\extramarks{}{Problem \arabic{#1} continued on next page\ldots}\nobreak{}
    \nobreak\extramarks{Problem \arabic{#1} (continued)}{Problem \arabic{#1} continued on next page\ldots}\nobreak{}
}

\newcommand{\exitProblemHeader}[1]{
    \nobreak\extramarks{Problem \arabic{#1} (continued)}{Problem \arabic{#1} continued on next page\ldots}\nobreak{}
    \stepcounter{#1}
    \nobreak\extramarks{Problem \arabic{#1}}{}\nobreak{}
}

\setcounter{secnumdepth}{0}
\newcounter{partCounter}
\newcounter{homeworkProblemCounter}
\setcounter{homeworkProblemCounter}{1}
\nobreak\extramarks{Problem \arabic{homeworkProblemCounter}}{}\nobreak{}

\newenvironment{homeworkProblem}[1][-1]{
    \ifnum#1>0
        \setcounter{homeworkProblemCounter}{#1}
    \fi
    \section{Problem \arabic{homeworkProblemCounter}}
    \setcounter{partCounter}{1}
    \enterProblemHeader{homeworkProblemCounter}
}{
    \exitProblemHeader{homeworkProblemCounter}
}

\renewcommand{\part}{\textbf{\\\\\large Part \Alph{partCounter}}\stepcounter{partCounter}\\} % part macro
\newcommand{\solution}{\textbf{\\\large Solution}\\} % solution macro

\newmdenv[
    backgroundcolor=gray!20,
    skipabove=\topsep,
    skipbelow=\topsep,
]{grayBoxed}

% General writing
\newcommand{\bracket}[1]{\left(#1\right)} % for automatic resizing of brackets
\newcommand{\sbracket}[1]{\left[#1\right]} % for automatic resizing of square brackets
\newcommand{\mset}[1]{\left\{#1\right\}} % for automatic resizing of curly brackets
\newcommand{\defeq}{\coloneqq} % the "defined as" command
\newcommand{\RNum}[1]{\uppercase\expandafter{\romannumeral #1\relax}} % uppercase roman numerals

\newcommand{\rednote}[1]{{\color{red} #1}} % for red text
\newcommand{\bluenote}[1]{{\color{blue} #1}} % for blue text
\newcommand{\greennote}[1]{{\color{green} #1}} % for green text

% Blackboard Maths Symbols
\newcommand{\N}{\mathbb{N}} % natural numbers
\newcommand{\Q}{\mathbb{Q}} % rational numbers
\newcommand{\Z}{\mathbb{Z}} % integers
\newcommand{\R}{\mathbb{R}} % real numbers
\newcommand{\C}{\mathbb{C}} % complex numbers
\newcommand{\E}{\mathbb{E}} % expectation operators

% Aesthetic

\newcommand{\defBox}[2]{
    \begin{tcolorbox}[colback=cyan!5!white,colframe=cyan!40!black,
        colbacktitle=cyan!40!black,title=Definition: #1]
        #2
    \end{tcolorbox}
}

\newcommand{\lemBox}[2]{
    \begin{tcolorbox}[colback=cyan!5!white,colframe=cyan!50!black,
        colbacktitle=cyan!50!black,title=Lemma: #1]
        #2
    \end{tcolorbox}
}

\newcommand{\thmBox}[2]{
    \begin{tcolorbox}[colback=cyan!5!white,colframe=cyan!70!black,
        colbacktitle=cyan!70!black,title=Theorem: #1]
        #2
    \end{tcolorbox}
}

% Homework Specific
\newcommand{\homBox}[1]{
    \begin{tcolorbox}[colback=cyan!5!white,colframe=cyan!50!black]
        #1
    \end{tcolorbox}
}
%%% DOCUMENT SETTINGS
\topmargin=-0.45in
\evensidemargin=0in
\oddsidemargin=0in
\textwidth=6.5in
\textheight=9.0in
\headsep=0.25in
\linespread{1.1}
\pagestyle{fancy}
\lhead{\hmwkAuthorName}
\chead{}
\rhead{\hmwkStudentnum}
\lfoot{\lastxmark}
\cfoot{\thepage}
\renewcommand\headrulewidth{0.4pt}
\renewcommand\footrulewidth{0.4pt}
\setlength\parindent{0pt}


%%% HOMEWORK DETAILS
\newcommand{\hmwkStudentnum}{22337763}
\newcommand{\hmwkClassInstructor}{LECTURER NAME AND TITLE}
\newcommand{\hmwkAuthorName}{\textbf{Viktor Ilchev}}


\title{
    \vspace{2in}
        % \textmd{\textbf{MT232P - Analysis}}\\
        % \textmd{\textbf{MT241P - Finite Mathematics}}\\
        \textmd{\textbf{MT251P - Foundations of Euclidean Geometry}}\\
    \vspace{1in}
    \textmd{\textbf{Assignment \#4}}\\
    \vspace{1in}
}

\author{
    \hmwkAuthorName\\
    \hmwkStudentnum
}

\date{}

\begin{document}

\maketitle

\pagebreak
\begin{homeworkProblem}
    \homBox{
        In each case below, state whether the statement is true or false. Justify your answer in
        each case.
    }
    \part
    \homBox{
        There are infinitely many 4 $\times$ 4 matrices that are not invertible.
    }
    \solution
    Since a square matrix is invertible if and only if its determinant is non-zero. Since there are infinitely many possible values for the elements of a 4 $\times$ 4 matrix, there are also infinitely many matrices that have a determinant of zero and are therefore not invertible. Therefore the statement is True
    \part
    \homBox{
        There is a 4 $\times$ 4 invertible matrix A such that $A^3 = 2A^2$ and $\det A = 2$.
    }
    \solution
    $$A^3 = 2A^2$$ $$A(AA) = 2(AA)$$ Since A is invertible, then $A^{-1}$ exists. So: $$A(AA)(A^{-1}A^{-1}) = 2(AA)(A^{-1}A^{-1})$$ $$A(AA^{-1})(AA^{-1}) = 2(AA^{-1})(AA^{-1})$$ We also know that $AA^{-1} = I_4$ since A is a $4\times 4$ matrix. So: $$A(I_4)(I_4) = 2(I_4)(I_4)$$ $$A = 2(I_4)$$ $$\det(A) = \det(2(I_4))$$ $$\det(A) = 16$$
    Since $16 \neq 2$, then the statement is False.
    \part
    \homBox{
        There is a 4 $\times$ 4 matrix A such that $A^2 =
            \begin{pmatrix}
                -1 & 0 & 0 & 0 \\
                0  & 1 & 0 & 0 \\
                0  & 0 & 1 & 0 \\
                0  & 0 & 0 & 1
            \end{pmatrix}$
    }
    \solution
    We see that the top left entry of $A^2$ is $-1$. Since the square of a number is always nonnegative and the entries have to be Real we have a contradiction.
    Therefore the statement is False.
\end{homeworkProblem}
\begin{homeworkProblem}
    \part
    \homBox{
        Prove that $\det (A^{-1}BA)$ = $\det (B)$, for all n $\times$ n matrices A, B, where A is invertible
        and n > 1.
    }
    \solution
    Let $A$ be an invertible $n \times n$ matrix, and let $B$ be any $n \times n$ matrix. We have
    $\det (A^{-1}BA) = \det (A^{-1}) \det (B) \det (A) = \frac{1}{\det (A)} \det (B) \det (A) = \det (B)$, so $\det (A^{-1}BA) = \det (B)$
    \part
    \homBox{
        Suppose $\underline{a} = i + 2j - k$, $\underline{b} = i + 3j + k$ and $\underline{c} = 3i + 8j + 4k$. Find $||\underline{w}||^2$ if $\underline{w} \in \R^3$ such that $\underline{w}.\underline{a} = 3$, $\underline{w}.\underline{b} = 5$ and $\underline{w}.\underline{c} = 17$.
    }
    \solution
    Since $\underline{w}$ is in $\R^3$, it must be represented in the form of $\underline{w} = xi + yj + zk$ for some $x, y, z \in \R$.

    We have the following:
    $$ \underline{w}.\underline{a} = (xi + yj + zk).(i + 2j - k) = x + 2y - z = 3 $$
    $$ \underline{w}.\underline{b} = (xi + yj + zk).(i + 3j + k) = x + 3y + z = 5 $$
    $$ \underline{w}.\underline{c} = (xi + yj + zk).(3i + 8j + 4k) = 3x + 8y + 4z = 17 $$Now we can solve for $x, y, z$ using a matrix:
    $$\begin{pmatrix}
            1 & 2 & -1 & 3  \\
            1 & 3 & 1  & 5  \\
            3 & 8 & 4  & 17
        \end{pmatrix}$$
    Replace $R_2$ with $R_2 - R_1$
    $$\begin{pmatrix}
            1 & 2 & -1 & 3  \\
            0 & 1 & 2  & 2  \\
            3 & 8 & 4  & 17
        \end{pmatrix}$$
    Replace $R_3$ with $R_3 - 3R_1$
    $$\begin{pmatrix}
            1 & 2 & -1 & 3 \\
            0 & 1 & 2  & 2 \\
            0 & 2 & 7  & 8
        \end{pmatrix}$$
    Replace $R_3$ with $R_3 - 2R_2$
    $$\begin{pmatrix}
            1 & 2 & -1 & 3 \\
            0 & 1 & 2  & 2 \\
            0 & 0 & 3  & 4
        \end{pmatrix}$$
    Replace $R_1$ with $R_1 - 2R_2$
    $$\begin{pmatrix}
            1 & 0 & -5 & -1 \\
            0 & 1 & 2  & 2  \\
            0 & 0 & 3  & 4
        \end{pmatrix}$$
    Replace $R_3$ with $\frac{1}{3}R_3$
    $$\begin{pmatrix}
            1 & 0 & -5 & -1          \\
            0 & 1 & 2  & 2           \\
            0 & 0 & 1  & \frac{4}{3}
        \end{pmatrix}$$
    Replace $R_2$ with $R_2 - 2R_3$
    $$\begin{pmatrix}
            1 & 0 & -5 & -1           \\
            0 & 1 & 0  & -\frac{2}{3} \\
            0 & 0 & 1  & \frac{4}{3}
        \end{pmatrix}$$
    Replace $R_1$ with $R_1 - (-5)R_3$
    $$\begin{pmatrix}
            1 & 0 & 0 & \frac{17}{3} \\
            0 & 1 & 0 & -\frac{2}{3} \\
            0 & 0 & 1 & \frac{4}{3}
        \end{pmatrix}$$
    Then: $x = \frac{17}{3}, y = -\frac{2}{3}, z = \frac{4}{3}$\\
    Therefore $||\underline{w}|| = \sqrt{x^2 + y^2 + z^2} = \sqrt{\frac{17}{3}^2 - \frac{2}{3}^2 + \frac{4}{3}^2} = \frac{\sqrt{309}}{3}$\\
    Then: $||\underline{w}||^2 = (\frac{\sqrt{309}}{3})^2 = \frac{103}{3}$
\end{homeworkProblem}
\begin{homeworkProblem}
    \homBox{
        Find the solution set of the following system of linear equations:
        $$x_1 - 4x_2 + 3x_3 = 0$$
        $$2x_1 - 6x_2 + 10x_3 = 6$$
        $$x_1 - 2x_2 + 7x_3 = 5$$
    }
    \solution
    We can solve for $x_1, x_2, x_3$ using a matrix:
    $$\begin{pmatrix}
            1 & -4 & 3  & 0 \\
            2 & -6 & 10 & 6 \\
            1 & -2 & 7  & 5
        \end{pmatrix}$$
    Replace $R_2$ with $R_2 - 2R_1$
    $$\begin{pmatrix}
            1 & -4 & 3 & 0 \\
            0 & 2  & 4 & 6 \\
            1 & -2 & 7 & 5
        \end{pmatrix}$$
    Replace $R_3$ with $R_3 - R_1$
    $$\begin{pmatrix}
            1 & -4 & 3 & 0 \\
            0 & 2  & 4 & 6 \\
            0 & 2  & 4 & 5
        \end{pmatrix}$$
    Replace $R_3$ with $R_3 - R_2$
    $$\begin{pmatrix}
            1 & -4 & 3 & 0  \\
            0 & 2  & 4 & 6  \\
            0 & 0  & 0 & -1
        \end{pmatrix}$$
    Notice how the last row states: $0 = -1$, therefore the system has no solutions.
\end{homeworkProblem}
\begin{homeworkProblem}
    \homBox{
        Find the solution set of the following system of linear equations:
        $$4x - 6y + 8z = 8$$
        $$x + 2y - 5z = 2$$
        $$y + 4x - 6z = 8$$
    }
    \solution
    We can solve for $x, y, z$ using a matrix:
    $$\begin{pmatrix}
            4 & -6 & 8  & 8 \\
            1 & 2  & -5 & 2 \\
            4 & 1  & -6 & 8
        \end{pmatrix}$$
    Replace $R_1$ with $R_2$
    $$\begin{pmatrix}
            1 & 2  & -5 & 2 \\
            4 & -6 & 8  & 8 \\
            4 & 1  & -6 & 8
        \end{pmatrix}$$
    Replace $R_2$ with $R_2 - 4R_1$
    $$\begin{pmatrix}
            1 & 2   & -5 & 2 \\
            0 & -14 & 28 & 0 \\
            4 & 1   & -6 & 8
        \end{pmatrix}$$
    Replace $R_3$ with $R_3 - 4R_1$
    $$\begin{pmatrix}
            1 & 2   & -5 & 2 \\
            0 & -14 & 28 & 0 \\
            0 & -7  & 14 & 0
        \end{pmatrix}$$
    Replace $R_2$ with $\frac{1}{2}R_2$
    $$\begin{pmatrix}
            1 & 2  & -5 & 2 \\
            0 & -7 & 14 & 0 \\
            0 & -7 & 14 & 0
        \end{pmatrix}$$
    Replace $R_3$ with $R_3 - R_2$
    $$\begin{pmatrix}
            1 & 2  & -5 & 2 \\
            0 & -7 & 14 & 0 \\
            0 & 0  & 0  & 0
        \end{pmatrix}$$
    Replace $R_2$ with $-\frac{1}{7}R_2$
    $$\begin{pmatrix}
            1 & 2 & -5 & 2 \\
            0 & 1 & -2 & 0 \\
            0 & 0 & 0  & 0
        \end{pmatrix}$$
    Notice how the last row states $0 = 0$, therefore the system has infinitely many solutions.\\
    Then the system of linear equations is: $$x + 2y - 5z = 2$$ $$y + -7z = 0$$
    By Remark 4, x, y are leading variables, and z is a free variable. We say that z = s, where s can be any real number.
    Then we can represent x, y in terms of z: $$ x = 2 - 2y + 5z = 2 - 2y + 5s $$ $$y = 7z = 7s$$ Then: $$x = 2 - 2y + 5s = 2 - 2(7s) + 5s = 2 - 9s$$
    Therefore the solution set of the system of equations is:$$\mset{(2 - 9s, 7s,s):s \in \R}$$
\end{homeworkProblem}
\begin{homeworkProblem}
    \homBox{
        Find $A^{-1}$ if $A =
            \begin{pmatrix}
                0 & 1  & 2 \\
                1 & 0  & 3 \\
                4 & -3 & 8
            \end{pmatrix}$
    }
    \solution
    $$A = \begin{pmatrix}
            0 & 1  & 2 \\
            1 & 0  & 3 \\
            4 & -3 & 8
        \end{pmatrix}
        \begin{pmatrix}
            1 & 0 & 0 \\
            0 & 1 & 0 \\
            0 & 0 & 1
        \end{pmatrix} = I_3$$
    Replace $R_1$ with $R_2$ in both matrices
    $$\begin{pmatrix}
            1 & 0  & 3 \\
            0 & 1  & 2 \\
            4 & -3 & 8
        \end{pmatrix}
        \begin{pmatrix}
            0 & 1 & 0 \\
            1 & 0 & 0 \\
            0 & 0 & 1
        \end{pmatrix}$$
    Replace $R_3$ with $R_3 - 4R_1$ in both matrices
    $$\begin{pmatrix}
            1 & 0  & 3  \\
            0 & 1  & 2  \\
            0 & -3 & -4
        \end{pmatrix}
        \begin{pmatrix}
            0 & 1  & 0 \\
            1 & 0  & 0 \\
            0 & -4 & 1
        \end{pmatrix}$$
    Replace $R_3$ with $R_3 - (-3)R_2$ in both matrices
    $$\begin{pmatrix}
            1 & 0 & 3 \\
            0 & 1 & 2 \\
            0 & 0 & 2
        \end{pmatrix}
        \begin{pmatrix}
            0 & 1  & 0 \\
            1 & 0  & 0 \\
            3 & -4 & 1
        \end{pmatrix}$$
    Replace $R_3$ with $\frac{1}{2}R_3$ in both matrices
    $$\begin{pmatrix}
            1 & 0 & 3 \\
            0 & 1 & 2 \\
            0 & 0 & 1
        \end{pmatrix}
        \begin{pmatrix}
            0           & 1  & 0           \\
            1           & 0  & 0           \\
            \frac{3}{2} & -2 & \frac{1}{2}
        \end{pmatrix}$$
    Replace $R_2$ with $R_2 - 2R_3$ in both matrices
    $$\begin{pmatrix}
            1 & 0 & 3 \\
            0 & 1 & 0 \\
            0 & 0 & 1
        \end{pmatrix}
        \begin{pmatrix}
            0           & 1  & 0           \\
            -2          & 4  & 1           \\
            \frac{3}{2} & -2 & \frac{1}{2}
        \end{pmatrix}$$
    Replace $R_1$ with $R_1 - 3R_3$ in both matrices
    $$\begin{pmatrix}
            1 & 0 & 0 \\
            0 & 1 & 0 \\
            0 & 0 & 1
        \end{pmatrix}
        \begin{pmatrix}
            -\frac{9}{2} & 7  & -\frac{3}{2} \\
            -2           & 4  & 1            \\
            \frac{3}{2}  & -2 & \frac{1}{2}
        \end{pmatrix}$$
    Since A is in RREF, then:
    $$A^{-1} = \begin{pmatrix}
        -\frac{9}{2} & 7  & -\frac{3}{2} \\
        -2           & 4  & 1            \\
        \frac{3}{2}  & -2 & \frac{1}{2}
    \end{pmatrix}$$
\end{homeworkProblem}
\end{document}
% \begin{homeworkProblem}
%     \homBox{Hello!}
%     \part
%     \solution
%     \begin{align}
%         Hello
%     \end{align}
%     \begin{align*}
%         Hello
%     \end{align*}
%     \begin{align}
%         Hello
%     \end{align}
%     Hello
%     \part
%     Hello\\Hello\\Hello
%     \solution
%     Hello
% \end{homeworkProblem}


% \begin{homeworkProblem}
%     \defBox{Hello}{
% \begin{itemize}
%     \item It will rain today or tomorrow.
%     \item It will rain today and tomorrow.
%     \item It will rain today but not tomorrow.
%     \item It will rain today or tomorrow, but not both days.
% \end{itemize}
%     }
%     \part
%     \solution

%     \part
%     \solution
% \end{homeworkProblem}


% \begin{homeworkProblem}
%     \lemBox{Hello}{

%     }
% \end{homeworkProblem}


% \begin{homeworkProblem}
%     \thmBox{Hello}{
%         \tcblower
%         Hello
%     }
% \end{homeworkProblem}