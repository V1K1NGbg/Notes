\renewcommand{\part}{\textbf{\\\large Part \Alph{partCounter}}\stepcounter{partCounter}\\} % part macro
\newcommand{\solution}{\textbf{\\\large Solution}\\} % solution macro

\newmdenv[
    backgroundcolor=gray!20,
    skipabove=\topsep,
    skipbelow=\topsep,
]{grayBoxed}

% General writing
\newcommand{\bracket}[1]{\left(#1\right)} % for automatic resizing of brackets
\newcommand{\sbracket}[1]{\left[#1\right]} % for automatic resizing of square brackets
\newcommand{\mset}[1]{\left\{#1\right\}} % for automatic resizing of curly brackets
\newcommand{\defeq}{\coloneqq} % the "defined as" command
\newcommand{\RNum}[1]{\uppercase\expandafter{\romannumeral #1\relax}} % uppercase roman numerals

\newcommand{\rednote}[1]{{\color{red} #1}} % for red text
\newcommand{\bluenote}[1]{{\color{blue} #1}} % for blue text
\newcommand{\greennote}[1]{{\color{green} #1}} % for green text

% Blackboard Maths Symbols
\newcommand{\N}{\mathbb{N}} % natural numbers
\newcommand{\Q}{\mathbb{Q}} % rational numbers
\newcommand{\Z}{\mathbb{Z}} % integers
\newcommand{\R}{\mathbb{R}} % real numbers
\newcommand{\C}{\mathbb{C}} % complex numbers
\newcommand{\E}{\mathbb{E}} % expectation operators

% Aesthetic

\newcommand{\defBox}[2]{
    \begin{tcolorbox}[colback=cyan!5!white,colframe=cyan!40!black,
        colbacktitle=cyan!40!black,title=Definition: #1]
        #2
    \end{tcolorbox}
}

\newcommand{\lemBox}[2]{
    \begin{tcolorbox}[colback=cyan!5!white,colframe=cyan!50!black,
        colbacktitle=cyan!50!black,title=Lemma: #1]
        #2
    \end{tcolorbox}
}

\newcommand{\thmBox}[2]{
    \begin{tcolorbox}[colback=cyan!5!white,colframe=cyan!70!black,
        colbacktitle=cyan!70!black,title=Theorem: #1]
        #2
    \end{tcolorbox}
}

% Homework Specific
\newcommand{\homBox}[1]{
    \begin{tcolorbox}[colback=cyan!5!white,colframe=cyan!50!black]
        #1
    \end{tcolorbox}
}